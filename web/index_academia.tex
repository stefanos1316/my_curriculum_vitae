% Copyright 2013 Christophe-Marie Duquesne <chmd@chmd.fr>
% Copyright 2014 Mark Szepieniec <http://github.com/mszep>
% 
% ConText style for making a resume with pandoc. Inspired by moderncv.
% 
% This CSS document is delivered to you under the CC BY-SA 3.0 License.
% https://creativecommons.org/licenses/by-sa/3.0/deed.en_US

\startmode[*mkii]
  \enableregime[utf-8]  
  \setupcolors[state=start]
\stopmode

\setupcolor[hex]
\definecolor[titlegrey][h=757575]
\definecolor[sectioncolor][h=397249]
\definecolor[rulecolor][h=9cb770]

% Enable hyperlinks
\setupinteraction[state=start, color=sectioncolor]

\setuppapersize [A4][A4]
\setuplayout    [width=middle, height=middle,
                 backspace=20mm, cutspace=0mm,
                 topspace=10mm, bottomspace=20mm,
                 header=0mm, footer=0mm]

%\setuppagenumbering[location={footer,center}]

\setupbodyfont[11pt, helvetica]

\setupwhitespace[medium]

\setupblackrules[width=31mm, color=rulecolor]

\setuphead[chapter]      [style=\tfd]
\setuphead[section]      [style=\tfd\bf, color=titlegrey, align=middle]
\setuphead[subsection]   [style=\tfb\bf, color=sectioncolor, align=right,
                          before={\leavevmode\blackrule\hspace}]
\setuphead[subsubsection][style=\bf]

\setuphead[chapter, section, subsection, subsubsection][number=no]

%\setupdescriptions[width=10mm]

\definedescription
  [description]
  [headstyle=bold, style=normal,
   location=hanging, width=18mm, distance=14mm, margin=0cm]

\setupitemize[autointro, packed]    % prevent orphan list intro
\setupitemize[indentnext=no]

\setupfloat[figure][default={here,nonumber}]
\setupfloat[table][default={here,nonumber}]

\setuptables[textwidth=max, HL=none]

\setupthinrules[width=15em] % width of horizontal rules

\setupdelimitedtext
  [blockquote]
  [before={\setupalign[middle]},
   indentnext=no,
  ]


\starttext

\startsectionlevel[title={Stefanos
Georgiou},reference={stefanos-georgiou}]

\thinrule

\startblockquote
{\bf Contact details}: first name 1316 at gmail dot com, sgeorgiou at
name at borocard dot com
\stopblockquote

\startblockquote
{\bf About me}: I am passionated with research,
\goto{coding}[url(https://stefanos1316.github.io/programmers_blog/)],
integrating new technologies, and automating cumbersome tasks. My
research interests lie to Green and Energy-Efficient Computing, Mining
Software Repositories, and Machine Learning. I enjoy working on the
command-line and especially with Linux systems. I love to create small
indepentant componets and tools to facilitate different daily
functionalities. On my free time, I enjoy sporting, reading books, and
\goto{travelling}[url(https://stefanos1316.github.io/my_blog/index.html)].
\stopblockquote

\thinrule

\startsectionlevel[title={Education},reference={education}]

\startdescription{2021-2022}
  {\bf PostDoctoral, Electrical and Computer Engineering}; Queen's
  University

  Lab name: Software Evolution and Analytics Lab

  Lab's head: \goto{Prof.~Ying (Jenny)
  Zou}[url(https://www.ece.queensu.ca/people/Y-Zou/)]
\stopdescription

\startdescription{2016-2021}
  {\bf PhD, Computer Science}; Athens University of Economics and
  Business

  Thesis title: {\em Energy and Run-Time Performance Practices in
  Software Engineering}; \goto{Thesis}[url(phd_thesis.pdf)]

  PhD advisors: (Main) \goto{Prof.~Diomidis
  Spinellis}[url(https://www2.dmst.aueb.gr/dds/)], \goto{Prof.~Panos
  Louridas}[url(https://www.balab.aueb.gr/panos-panagiotis-louridas.html)],
  and \goto{Rizos Sakellariou}[url(http://www.cs.man.ac.uk/~rizos/)]
\stopdescription

\startdescription{2013-2015}
  {\bf MSc, Pervasive Computing and Communications for Sustainable
  Development (PERCCOM)}; Erasmus Mundus Joint Master Degree

  ITMO University of Saint-Petersburg (Russia) Thesis title: {\em
  Implementating Green IT Approach for Transferring Big Data over
  Parallel Data Links}; \goto{Thesis}[url(msc_thesis.pdf)]

  Lulea University of Technology (Sweden) Semester 3: {\em Resource
  Efficient Pervasive Computing Systems and Communications}
  \goto{Degree}[url(msc_lulea.pdf)]

  Lappeenranta University of Technology (Finland) Semester 2: {\em Smart
  Software and Services} \goto{Degree}[url(msc_lappeenranta.pdf)]

  Unversity of Lorraine (France) Semester 1: {\em Sustainable Computer
  Network Engineering} \goto{Degree}[url(msc_lorraine.pdf)]
\stopdescription

\startdescription{2008-2013}
  {\bf BSc, Networks and Systems Programming}; University of Cyprus

  Thesis title: {\em Implementation and Evaluation of the Biologically
  -- Inspired AntHocNet Routing Protocol in Sensor Network};
  \goto{Degree}[url(bsc_cyprus.pdf)]
\stopdescription

\stopsectionlevel

\startsectionlevel[title={Research
Publications},reference={research-publications}]

\startitemize[packed]
\item
  Conference Papers
  \startitemize[packed]
  \item
    {\bf Stefanos Georgiou}*, Maria Kechagia, Tushar Sharma, Federica
    Sarro, and Jenny Zou. Green AI: Do Deep Learning Frameworks Have
    Different Costs? In 44th International Conference on Software
    Engineering: Technical Track, ICSE '22. New York, NY, USA, May 2022.
    Association for Computing Machinery.
    {[}\goto{Paper}[url(GKSSZ22.pdf)]{]}
    {[}\goto{BibTex}[url(GKSSZ22.bib)]{]}
  \item
    {\bf Stefanos Georgiou}, Maria Kechagia Panos Louridas, and Diomidis
    Spinellis. What Are Your Programming Language's Energy-Delay
    Implications? In 15th International Conference on Mining Software
    Repositories: Technical Track, MSR '18. New York, NY, USA, May 2018.
    Association for Computing Machinery.
    {[}\goto{Paper}[url(GKLS18.pdf)]{]}
    {[}\goto{BibTex}[url(GKLS18.bib)]{]}
    {[}\goto{Presentation}[url(https://www.slideshare.net/GeorgiouStefanos/what-are-your-programming-languages-energydelay-implications-106251480)]{]}
    {[}\goto{Data-set}[url(https://github.com/stefanos1316/Rosetta_Code_Data_Set)]{]}
  \item
    {\bf Stefanos Georgiou}, Maria Kechagia, and Diomidis Spinellis.
    Analyzing Programming Languages' Energy Consumption: An Empirical
    Study. In PCI 2017: Proceedings of the 21st Pan-Hellenic Conference
    on Informatics, ACM International Conference Proceeding Series. ACM
    Press, September 2017. {[}\goto{Paper}[url(GKS17.pdf)]{]}
    {[}\goto{BibTex}[url(GKS17.bib)]{]}
    {[}\goto{Presentation}[url(https://www.slideshare.net/GeorgiouStefanos/programming-languages-energy-consumption-an-empirical-study)]{]}
    {[}\goto{Data-Set}[url(https://github.com/stefanos1316/Rosetta_Code_Data_Set)]{]}
  \stopitemize
\item
  Journal Articles
  \startitemize[packed]
  \item
    {\bf Stefanos Georgiou}, Dimitris Mitropoulos, and Diomidis
    Spinellis. Energy-Efficient Computing in a Secure Environment.
    Submitted in July 2020 for review in ACM Transaction of Computer
    Systems.
  \item
    {\bf Stefanos Georgiou} and Diomidis Spinellis. Energy-Delay
    Investigation of Remote Inter-Process Communication Technologies.
    Accepted for publication in Elsevier Journal of Systems and Software
    2019. {[}\goto{Preprint}[url(GS19.pdf)]{]}
    {[}\goto{BibTex}[url(GS19.bib)]{]}
  \item
    {\bf Stefanos Georgiou}, Stamatia Rizou, and Diomidis Spinellis.
    Software Development Life Cycle for Energy-Efficiency: Techniques
    and Tools. Accepted for publication in ACM Computing Surveys 2019.
    {[}\goto{Preprint}[url(GRS19.pdf)]{]}
    {[}\goto{BibTex}[url(GRS19.bib)]{]}
  \stopitemize
\stopitemize

\stopsectionlevel

\startsectionlevel[title={Organization
Services},reference={organization-services}]

\startdescription{{\em Program Committee}}
  EASE 2022 for the Research Track

  ISEC 2022 for the Doctoral Symposium

  ESEC/FSE 2021 for the Student Research Competition

  ASE 2020 for the International Workshop on Sustainable Software
  Engineering
\stopdescription

\startdescription{{\em Reviewer}}
  Journals submitted to Software: Practices and Experience

  Journals submitted to ACM Computing Surveys (CSUR)

  Research papers submitted to ESEC/FSE
\stopdescription

\startdescription{{\em Co-reviewer}}
  Conference papers submitted to ICSE-SEIP, MSR, OSS, ICPC, ICCS, SANER,
  FSE, ASE, ICMSE and SATToSE
\stopdescription

\stopsectionlevel

\startsectionlevel[title={Invited Talks},reference={invited-talks}]

\startdescription{Tutorials}
  {\em How to improve your CI/CD process} presented on 5th of July 2019
  at BALab of Athens University of Economics and Business in Athens,
  Greece.
  \goto{Presentation}[url(https://aueb-balab.github.io/courses/tools/ci_cd_with_gitlab-p.html\#/)]

  {\em Travis CI with CV template tutorial} presented on 7th of June
  2018 at ITMO Univesity of Saint Petersburg for PERCCOM Master students
  in Saint-Petersburg, Russia.
  \goto{Presentation}[url(https://aueb-balab.github.io/courses/tools/travis_ci_cv_template-p.html\#/)]
\stopdescription

\startdescription{Research}
  {\em Saving Energy in Software Development by Making the Right
  Choices} presented on the 4th of March 2022 at Sustainable Software
  Engineering lecture of TU Delft.
  \goto{Presentation}[url(https://tudelft.zoom.us/rec/play/5hMydhFhbQjzQonYH6CuSdD3iIGGd6nqEFmYbuT8gd_-0c0uS9L5cgzCbtHnW1bhbE5ZBUP_Wi5tsu2-.a5o7ggYccw7UJ65R?continueMode=true&_x_zm_rtaid=vsffdTXoSwOo6_1ITMqcVg.1646419616427.8eaf7f4a43123fd65c05f4005b1c483d&_x_zm_rhtaid=587)]
  \goto{Slides}[url(https://docs.google.com/presentation/d/1mywyQ8ydxHYrbRYdSJhAiQnJnSYD3DXe/edit\#slide=id.p1)]

  {\em What are your programming language's energy-delay implications?}
  presented on the 12th of June 2018 at PERCCOM's Summer School in
  Lappeenranta, Finland.
  \goto{Proof}[url(perccom_summer_school_2018.pdf)]
\stopdescription

\stopsectionlevel

\startsectionlevel[title={Teaching
Experience},reference={teaching-experience}]

Having weekly meetings with five PhD students and two Master students to
assist and advise them on their research studies (at Queen's University
from 2021 to 2023)

Assisting my PhD advisor at Programming II course (Winter Semester of
2016, 2017, and 2018) by giving tutorial and Lab sessions in Java
{[}\goto{Repository}[url(https://github.com/AUEB-BALab/courses)]{]}

Gave Java lectures and tutorials for the 1st and 2nd Coding Boot Camp in
Athens, Greece (Oct.~2017 - May 2018)
{[}\goto{Repository}[url(https://github.com/codeandwork/courses)]{]}
{[}\goto{Proof}[url(boot_camp_athens_2016_2017.pdf)]{]}

\stopsectionlevel

\startsectionlevel[title={Experience from
Industry},reference={experience-from-industry}]

\startdescription{Feb 2021-currently}
  {\bf Back-end developer and DevOps:}
  \goto{Boro}[url(http://www.borocard.com/)]

  Coordinating the development of a savings management mobile
  application.

  \startdescription{{\em Responsibilities}}
    Developing server via NodeJS and MongoDB to host appliaction and
    manage its clients.

    Created the CI pipeline of the back-end repository.

    Manage the AWS hosting of the server.

    Established all pre-commit hooks to check commits source code
    quality.
  \stopdescription
\stopdescription

\startdescription{Sep 2020-Feb 2021}
  {\bf Software Developer} \goto{Greek Free, Open Source
  Software}[url(https://gfoss.eu/)]

  Developing a privacy-preserving epidemic dosimeter based on contact
  tracing. \goto{Proof}[url(gfoss_2020.pdf)]
  \goto{Repository}[url(https://github.com/eellak/epidose.git)]

  \startdescription{{\em Responsibilities}}
    Developing \goto{Python}[url(https://www.python.org/)],
    \goto{Shell}[url(https://www.gnu.org/software/bash/)], and
    \goto{Ansible}[url(https://www.ansible.com/)] scripts on a Raspberry
    Pi Zero

    Extented testing for the correct functionality of the device
  \stopdescription
\stopdescription

\startdescription{Jan 2019-Apr 2020}
  {\bf Back-end developer, DevOps, and Integrations:} \goto{AllCanCode
  Inc.~Greek Branch}[url(https://www.allcancode.com/)]

  Supported the product development that facilitates fast web-sites
  development (by using
  \goto{Blockly}[url(https://developers.google.com/blockly)]) on Desktop
  that can be exported in smart-phones as well.

  \startdescription{{\em Responsibilities}}
    Development of API end-points in JavaScript
    (\goto{Node.JS}[url(https://nodejs.org/en/)]) and
    \goto{MongoDB}[url(https://www.mongodb.com/)] for the server
    platform (product) and customer products

    Creation of the CI system for the product (in GitLab) to perform
    back-end and front-end testing (through
    \goto{Mocha}[url(https://mochajs.org/)] and
    \goto{Cypress}[url(https://www.cypress.io/)] frameworks,
    respectively) and CD to auto-deploy product (for specific push
    branches) on Google's App Engine (Aurora, Beta, and Production)
    using the \goto{Flex
    environment}[url(https://cloud.google.com/appengine/docs/flexible)]

    Migration of monolithic product to micro-services using
    \goto{Firebase
    hosting}[url(https://firebase.google.com/docs/hosting)], \goto{Cloud
    Functions}[url(https://firebase.google.com/docs/functions)], and
    \goto{Cloud Run}[url(https://cloud.google.com/run)]

    Integration of \goto{Bitrise}[url(https://www.bitrise.io/)] system
    in the product to allow platform users in exporting their Desktop
    applications to Android and iOS smart-phones through the
    \goto{Cordova}[url(https://cordova.apache.org/)] wrapper
  \stopdescription
\stopdescription

\startdescription{Jan 2016-2019}
  {\bf Proposals writing:} \goto{Singular Logic
  S.A.}[url(https://portal.singularlogic.eu/en)]

  Proposals writing for the European Projects Department.
  \goto{Proof}[url(singular_logic_2016-2019.pdf)]

  \startdescription{{\em Responsibilities}}
    Writing research proposals for Horizon 2020
  \stopdescription
\stopdescription

\stopsectionlevel

\startsectionlevel[title={Technical
Experience},reference={technical-experience}]

\startdescription{{\em Back-End}}
  BASH and SHELL (Often automating cumbersome and time consuming tasks)

  NodeJS (Fan of KOA and Express, prefer Mocha-Chai, like to use Native
  Addons for performance)

  Java (Was also teaching it at the Athens University of Economics and
  Business)

  Python (No need to read tutorials, feeling confy to code anytime)

  C (First language to learn and impressed by its performance)

  C++ (Second to learn, but as soon as I learned Java I stopped using
  it)

  MongoDB (\goto{Basics}[url(mongodb_m001,pdf)],
  \goto{Aggregations}[url(mongodb_m121.pdf)],
  \goto{JavaScript}[url(mongodb_m220.pdf)], and
  \goto{Performance}[url(mongodb_m201.pdf)])
\stopdescription

\startdescription{{\em DevOps}}
  Continuous Integration and Deployment (Big fan of CI/CD Travis,
  GitLab, and GitHub Actions)

  Code Quality (Coveralls, maven-plugins, Style-Checkers, Reports,
  Prettier, and pre-commit hooks)

  Hosting (Firebase, AWS, and Google App Engine)

  Mobile CI/CD (Bitrise with Cordova wrapper for mobile apps)

  Configuration Management (Ansible because it uses YAML)
\stopdescription

\startdescription{{\em Projects}}
  Epidose: A privacy-preserving epidemic dosimeter based on contact
  tracing \goto{Repository}[url(https://github.com/eellak/epidose)]

  Exam Questionnaire Scanner:
  \goto{Repository}[url(https://github.com/AntonisGkortzis/ExamQuestionnaireScanner)]

  Measuring Energy Consumption:
  \goto{Software}[url(https://github.com/stefanos1316/SEMTs_Comparisson)]
  and
  \goto{Hardware}[url(https://stefanos1316.github.io/courses/tools/measuring_energy_consumption_direct_approach-p.html\#/)]
  tools

  Programming II website:
  \goto{Repository}[url(https://github.com/stefanos1316/courses)]

  Rosetta Code Experiement:
  \goto{Repository}[url(https://github.com/stefanos1316/Rosetta_Code_Data_Set)]

  Validate Links:
  \goto{Repository}[url(https://github.com/stefanos1316/validateLinks)]
\stopdescription

\startdescription{{\em Certificates}}
  Unix Tools: \goto{Data, Software, and Production
  Engineering}[url(unix_tools_edx.pdf)]
\stopdescription

\startdescription{{\em Course Completion}}
  MongoDB: \goto{Basics}[url(mongodb_m001.pdf)],
  \goto{Aggregation}[url(mongodb_m121.pdf)],
  \goto{JavaScript}[url(mongodb_m220.pdf)], and
  \goto{Performance}[url(mongodb_m201.pdf)]
\stopdescription

\starttyping
Loading the data just for you.
\stoptyping

\stopsectionlevel

\startsectionlevel[title={Grands, Awards, Achievements, Languages, and
Living
Abroad},reference={grands-awards-achievements-languages-and-living-abroad}]

\startitemize[packed]
\item
  Grands and awards:
  \startitemize[packed]
  \item
    \goto{Arctic Code Vault
    Contributor}[url(arctic_code_contributor.jpg)]
  \item
    Mentored younger researcher \goto{who won 2nd place}[url(fse19.pdf)]
    in SRC ESEC/FSE '19.
  \item
    Best paper award from the 15th Annual DMST Student Conference (2018)
  \item
    Marie Skłodowska-Curie funds,
    \goto{SENECA}[url(https://portal.singularlogic.eu/en/eu-project/12374/seneca)]
    (2016-2019)
  \item
    Erasmus Mundus Masters Scholarship, PERCCOM (2013-2015)
  \stopitemize
\item
  Human Languages:
  \startitemize[packed]
  \item
    Greek/Hungarian (Native speaker)
  \item
    English (Fluent speaker)
  \item
    Russian/France (Basic knowledge)
  \stopitemize
\item
  Living Abroad:
  \startitemize[packed]
  \item
    Born in Gyula, Hungary but grown up in Nicosia, Cyprus.
  \item
    Ate baguettes and drunk wine in Lorraine, France (Fall Semester of
    2013)
  \item
    Attended to many sauna parties in Lappeenranta, Finland (Spring
    Semester of 2014)
  \item
    Fish ice-cold beers from lakes in Lulea, Sweden (Fall Semeseter of
    2014)
  \item
    Drunk vodka and ate borsh in Saint-Petersburg, Russia (Spring
    Semester of 2015)
  \item
    Teaching in Athens and visiting islands in Greece (Fall Semester of
    2016 till now)
  \item
    Biked in Delft, the Netherlands for my PhD secondment (Half month of
    Sept.~2017)
  \stopitemize
\stopitemize

\stopsectionlevel

\startsectionlevel[title={My Academic Genealogy
Tree},reference={my-academic-genealogy-tree}]

Zoom-in for more details
\thinrule

\stopsectionlevel

\stopsectionlevel

\stoptext
